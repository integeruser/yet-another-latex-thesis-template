\chapter{Abstract}
The effectiveness of debugging software issues largely depends on the capabilities of the tools available to aid in such task. At present, to debug the macOS kernel there are no alternatives other than the basic debugger integrated in the kernel itself or the GDB stub implemented in VMware Fusion. However, due to design constraints and implementation choices, both approaches have several drawbacks, such as the lack of hardware breakpoints and the capability of pausing the execution of the kernel from the debugger, or the inadequate performance of the GDB stub for some debugging tasks.

The aim of this work is to improve the overall debugging experience of the macOS kernel, and to this end LLDBagility has been developed. This tool enables kernel debugging via \glsentrylong{vmi}, allowing to connect the LLDB debugger to any unmodified macOS \glsentrylong{vm} running on a patched version of the VirtualBox hypervisor. This solution overcomes all the limitations of the other debugging methods, and also implements new useful features, such as saving and restoring the state of the \glsentrylong{vm} directly from the debugger. In addition, a technique for using the lldbmacros debug scripts while debugging kernel builds that lack debug information is provided. As a case study, the proposed solution is evaluated on a typical kernel debugging session.
